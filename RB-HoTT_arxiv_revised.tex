\pdfoutput=1
\documentclass[11pt]{article}

% --- Packages ---
\usepackage[utf8]{inputenc}
\usepackage[T1]{fontenc}
\usepackage{lmodern}
\usepackage{microtype}
\usepackage[a4paper,margin=1in]{geometry}
\usepackage{mathtools,amssymb,amsmath,amsthm}
\usepackage{mathpartir} % for typing rules
\usepackage{stmaryrd}   % \llbracket\rrbracket
\usepackage{enumitem}
\usepackage{hyperref}
\usepackage[nameinlink,capitalize]{cleveref}
\usepackage{tikz-cd}
\usepackage{xcolor}
\usepackage{listings}
\lstset{basicstyle=\ttfamily\small,breaklines=true,columns=fullflexible}

% --- Theorem environments ---
\newtheorem{theorem}{Theorem}
\newtheorem{lemma}{Lemma}
\newtheorem{definition}{Definition}
\newtheorem{remark}{Remark}
\newtheorem{example}{Example}

% --- Macros ---
\newcommand{\Res}{\mathsf{Res}}
\newcommand{\Time}{\mathsf{time}}
\newcommand{\Mem}{\mathsf{mem}}
\newcommand{\Depth}{\mathsf{depth}}
\newcommand{\Nat}{\mathbb{N}}
\newcommand{\seq}{\Rightarrow^{*}}
\newcommand{\bb}{\mathsf{b}}
\newcommand{\ctx}{\Gamma}
\newcommand{\Id}{\mathsf{Id}}
\newcommand{\UR}{\mathcal{U}}
\newcommand{\modal}[1]{\Box_{#1}}
\newcommand{\leqR}{\preceq}
\newcommand{\plus}{\oplus}

% --- Title ---
\title{Resource-Bounded Homotopy Type Theory (RB-HoTT):\\Proof-Carrying Resources for Mathematics and Programs}
\author{RB-HoTT Team (Draft)}
\date{September 15, 2025}

\begin{document}
\maketitle

\begin{abstract}
\textbf{Resource-Bounded Homotopy Type Theory (RB-HoTT)} extends type theory to certify not only correctness but also computational costs—worst-case time, memory, and depth. Traditional formal verification proves that programs are correct but says little about whether they meet performance budgets. RB-HoTT makes resource bounds first-class: types carry explicit budgets, and proofs certify both functional correctness and resource compliance. We present a calculus with compositional bound synthesis, validate it via presheaf semantics, and demonstrate integration with Lean 4/mathlib through CI-enforced budget gates. This enables \emph{proof-carrying resources} for safety-critical systems, blockchain contracts, and embedded devices where exceeding budgets means failure.
\end{abstract}

\section{Purpose: Why Resource-Bounded Foundations?}
Formal methods typically certify \emph{what holds}, not \emph{what it costs}. In many settings—safety-critical controllers, on-device ML, blockchain contracts, and time-bounded autonomous agents—a proof with no budget is insufficient: a correct component that exceeds its time/memory envelope is unusable. RB-HoTT makes boundedness a first-class citizen alongside logical validity: the same certificate that proves a claim also attests that it fits within an explicit budget.\\
\textbf{Thesis (proof-carrying resources).} For a typed term $t$ (program/proof), we synthesize a bound $\bb$ and check an execution cost $k$ so that
\[
\ctx \vdash_{R;\,\bb} t : A \quad\Longrightarrow\quad \exists v.\; t \seq v\;\wedge\; k\le \bb \le \Time(R).
\]
Intuitively, the same artifact that certifies semantics also certifies a worst-case cost, ordered by a resource context $R$.

\subsection{Motivating Examples}
\begin{description}[leftmargin=2em,labelsep=0.5em]
  \item[Safety-critical real-time:] An automotive braking system must not only compute the correct braking force but do so within $10$\,ms.
  \item[Blockchain gas:] Smart contracts execute under strict gas limits; a functionally correct contract that exceeds gas bounds fails to execute.
  \item[Edge/embedded ML:] A mobile model that needs $8$\,GB RAM on a $4$\,GB device is unusable despite high accuracy.
\end{description}

\subsection{The RB-HoTT Solution}
RB-HoTT internalizes resource bounds via (i) a compositional bound-synthesis discipline, (ii) a feasibility modality $\modal{R}$ \emph{as a graded comonad} (interior operator), and (iii) presheaf semantics over a resource category. The resulting proofs certify both correctness and resource conformance.

\subsection{Contributions}
\begin{enumerate}[leftmargin=2em]
  \item A compositional calculus for resource-bounded terms (\S3).
  \item A left-exact feasibility modality with presheaf semantics (\S4).
  \item A dependent layer with resource-indexed universes (\S5).
  \item An engineering integration for Lean 4 with CI enforcement (\S6).
  \item A roadmap toward HoTT features (univalence, HITs) under budgets (\S7--9).
\end{enumerate}

\section{Philosophy and Design Principles}
\label{sec:principles}
\begin{enumerate}[leftmargin=*,label=\textbf{P\arabic*.}]
  \item \textbf{Budgets are first-class.} Resources appear in judgments and types.
  \item \textbf{Local, compositional rules.} Bound synthesis mirrors term structure.
  \item \textbf{Monotonicity/weakening.} If $R\leqR R'$, then $\modal{R}A \to \modal{R'}A$.
  \item \textbf{Left-exact, comonadic modality.} $\modal{R}$ acts as a graded \emph{interior} operator (coeffect/comonad), preserving finite limits and interacting predictably with $\Pi,\Sigma,\Id$.
  \item \textbf{Semantics first, kernel last.} Validate rules in presheaves; avoid kernel changes initially.
  \item \textbf{Pragmatism.} Start set-/groupoid-level; add higher structure and univalence where \emph{feasible}.
\end{enumerate}

\paragraph{Notation.} We use $\Res$ for resource contexts with pointwise order $\preceq$ and monoidal sum $\oplus$; $\Box_R A$ denotes availability at budget $R$; $\bb$ is a synthesized bound; $t\Rightarrow^{*}v$ (written $\seq$) is multi-step reduction.

\section{A Minimal Calculus for RB-HoTT}
We begin with a simply-typed core with costs, a resource algebra, and a feasibility modality; dependent types and HoTT features follow in \S\ref{sec:dependent}.

\subsection{Resource Contexts}
\begin{definition}[Resource algebra]
A resource context is a triple $R=(\Time,\Mem,\Depth)\in\Nat^3$ with pointwise preorder $\leqR$ and monoidal sum
\(
(t,m,d)\plus(t',m',d') := (t{+}t',\, m{+}m',\, \max(d,d')).
\)
\end{definition}
We require $(\Res,\plus,0)$ to be a commutative monoid and $\leqR$ to be compatible with $\plus$ (monotone in both arguments).

\paragraph{Extended algebras.} Beyond $(\Time,\Mem,\Depth)$ we admit parameterized resource algebras $(\Res,\preceq,\oplus,0)$ with application-specific dimensions (e.g., bandwidth, energy, gas), provided $\oplus$ is commutative/associative with identity and monotone in both arguments. \emph{Examples:} $(\Time,\Mem,\mathrm{bw},\mathrm{lat})$, $(\Time,\Mem,\mathrm{energy},\mathrm{heat})$, $(\mathrm{gas},\mathrm{storage},\mathrm{cycles})$.

\subsection{Typing with Synthesized Bounds (STLC)}
Judgments carry both a budgeted context and a synthesized bound:
\(
\ctx \vdash_{R;\,\bb} t:A.
\)
Composition is structural; e.g. application and pairing:
\begin{mathpar}
\inferrule{\ctx \vdash_{R;\,b_f} f: A\to B \\ \ctx \vdash_{R;\,b_a} a:A}
          {\ctx \vdash_{R;\,b_f+b_a+1} f\,a:B}
\qquad
\inferrule{\ctx \vdash_{R;\,b_a} a:A \\ \ctx \vdash_{R;\,b_b} b:B}
          {\ctx \vdash_{R;\,b_a+b_b} (a,b):A\times B}
\end{mathpar}
\paragraph{Conditionals.} We take the maximum of branch bounds plus the decision cost:
\begin{mathpar}
\inferrule{\ctx\vdash_{R;\,b_c} c: \mathsf{Bool} \\ \ctx\vdash_{R;\,b_t} t:A \\ \ctx\vdash_{R;\,b_f} f:A}
          {\ctx\vdash_{R;\,b_c+\max(b_t,b_f)+1} \mathbf{if}\ c\ \mathbf{then}\ t\ \mathbf{else}\ f : A}
\end{mathpar}
\paragraph{Recursion.} We model recursion via a depth budget. A safe rule (to be validated by a fuel/well-founded encoding) is:
\begin{mathpar}
\inferrule{\ctx, f{:}A\to B, x{:}A \vdash_{R;\,b} t:B \\ \Depth(R)>0}
          {\ctx \vdash_{R;\,\Depth(R)\cdot b} \mathsf{fix}\ f.\lambda x.t : A\to B}
\end{mathpar}
\begin{remark}[Implementing recursion safely]
We use a \emph{fuelled} or well-founded recursor in the implementation and \emph{prove} the multiplicative bound $\Depth(R)\cdot b$ as a lemma. This avoids unsound primitive rules while retaining the intended complexity guarantee.
\end{remark}

\subsection{Operational Cost and Soundness}
We adopt a unit-step operational cost.
\begin{definition}[Operational cost]
Each core reduction carries unit cost, e.g.
$$(\lambda x.t)\ v \to t[v/x] \;[1], \qquad \mathbf{if}\;\mathsf{true}\;\mathbf{then}\;t\;\mathbf{else}\;f \to t \;[1].$$
We write $t \Downarrow_k v$ for ``$t$ reduces to $v$ in exactly $k$ steps'', and $t\seq v$ for some $k$.
\end{definition}
Small-step $t\to t'$ yields $t\seq v$ with unit step-cost. We also write $t \Downarrow_k v$ to mean ``$t$ reduces to $v$ in exactly $k$ steps'' and use $t\seq v$ when the exact $k$ is irrelevant.
\begin{theorem}[Cost soundness]
If $\ctx \vdash_{R;\,\bb} t:A$ and $t$ is closed, then $\exists v,k.\ t\seq v$ with $k\le \bb \le \Time(R)$.
\end{theorem}

\subsection{Availability / Feasibility Modality as Interior (Graded Comonad)}
\label{sec:box}
We treat $\modal{R}$ as a \emph{graded comonad} (interior operator) indexed by $R\in\Res$.
\paragraph{Core rules.}
\begin{mathpar}
\inferrule{}{\modal{R}A \to A}\quad(\textit{counit } \varepsilon)\qquad
\inferrule{}{\modal{R_1\oplus R_2}A \to \modal{R_1}\modal{R_2}A}\quad(\textit{comultiplication } \delta)\\
\inferrule{R\leqR R'}{\modal{R}A \to \modal{R'}A}\quad(\textit{monotonicity})
\end{mathpar}
\paragraph{Cost-aware boxing (introduction).} Promotion is \emph{not} unconditional. If $\ctx\vdash_{R;\,b} t:A$ with $b\le \Time(R)$, we may form
\[
\mathrm{box}_R(t) : \modal{R}A.
\]
Intuitively, you can only obtain $\Box_R A$ by actually constructing $A$ within budget $R$; there is no general rule $A\to\Box_R A$.
\paragraph{Advanced modality rules (schematic).}
\begin{mathpar}
\inferrule{\modal{R}A \;\;\; \modal{R}B}{\modal{R}(A\times B)}\qquad
\inferrule{\modal{R_1}A \;\;\; \modal{R_2}B}{\modal{R_1\oplus R_2}(A\times B)}
\end{mathpar}
Products are handled via $\delta$ and pairing; similar left-exactness holds for pullbacks.

\section{Presheaf Semantics over Resources (Shift Model)}
\label{sec:semantics}
Let $\Res$ be the thin category with objects resource contexts and a unique morphism $R'\to R$ iff $R'\leqR R$. A \emph{presheaf} $F: \Res^{op} \to \mathbf{Set}$ assigns each $R$ a set $F(R)$ (inhabitants feasible at budget $R$) and for $R'\leqR R$ a restriction $F(R)\to F(R')$ satisfying functoriality.
\paragraph{Concrete presheaves.} $\widehat{\mathbb{N}}(R)=\{n\in\Nat\mid n\le\Time(R)\}$ with clamping; $\mathsf{Bool}$ as a constant presheaf.
\paragraph{Interpreting $\modal{R}$ by resource shift.}
\emph{Shift model:}
\[
(\modal{R}A)(S) := A(S\oplus R).
\]
Because $S\leqR S\oplus R$, contravariance yields $\varepsilon: A(S\oplus R)\to A(S)$ (counit). Associativity of $\oplus$ yields $\delta: A(S\oplus(R_1\oplus R_2)) \cong A((S\oplus R_1)\oplus R_2)$ (comultiplication). Monotonicity is induced by monotonicity of $\oplus$.
\paragraph{Soundness sketch.} Interpret judgments as sections with cost witnesses; compositionality of rules corresponds to naturality and the monoid laws on $\Res$, yielding Theorem~1.

\subsection{Running Example: Binary Search (Worked)}
We illustrate bound synthesis end-to-end on binary search.
\begin{lstlisting}[language=lean]
-- Pseudocode sketch (Lean-like):
structure Res := (time mem depth : Nat)
notation R:65 " ⊕ " R':65 => 
  Res.mk (R.time+R'.time) (R.mem+R'.mem) (max R.depth R'.depth)

abbrev R_bs (n : Nat) : Res := ⟨Nat.ceil (log2 n) + 5, n, Nat.log2 n⟩

axiom binarySearch : (arr : Array α) → (target : α) → □(R_bs arr.size) (Option Nat)
axiom binarySearch_correct : ...
axiom binarySearch_bounded : ...
\end{lstlisting}
Application contributes $+1$; branching adds decision cost and $\max$ of branches; recursion consumes depth fuel as in \S3.

\section{Dependent Layer and HoTT Features}
\label{sec:dependent}
We extend with $\Pi,\Sigma,\Id$ and resource-indexed universes $\UR_R$ with cumulativity along $\leqR$. Formation/intro/elimination rules carry bounds that compose additively. Identity types use a $J$-rule whose side-conditions account for transport cost. Univalence and HITs are optional and gated by explicit feasibility predicates.
\paragraph{Resource-indexed universes.} Postulate $\UR_R\subseteq\UR_{R'}$ when $R\leqR R'$ and membership side-conditions $\bb\le \mathsf{complexity}(R)$ for codes inhabiting $\UR_R$.
\paragraph{Complexity predicate.} Choose $\mathsf{complexity}:\Res\to\mathbb{N}$ (default $=\Time$); richer models may combine time/memory or include overheads for codes.
\paragraph{Feasible univalence.} Admit $ua(f):\Id_{\UR_R}(A,B)$ from an equivalence $f:A\simeq B$ when $R$ satisfies explicit univalence budgets; charge a fixed $c_{\mathrm{univ}}$ to bounds.

\section{Engineering in Lean/mathlib: Proof-Carrying Resources}
\label{sec:engineering}
We instrument elaborated proof terms with structural proxies and enforce budgets in CI.
\paragraph{Structural cost model.} For term $t$, $\mathsf{cost}(t)=\alpha\cdot\mathsf{nodes}(t)+\beta\cdot\mathsf{lamDepth}(t)+\gamma\cdot\mathsf{apps}(t)+\delta\cdot\mathsf{cases}(t)$ with per-platform calibration.
\paragraph{Complete example (wrapper).}
\begin{lstlisting}[language=lean]
namespace RBHoTT.List

@[simp] def length {α} (l : List α) : Nat := List.length l

-- theorem length_bound (l : List α) : cost (length l) ≤ l.length + 3 := by
--   unfold length; admit
end RBHoTT.List
\end{lstlisting}
\paragraph{CI gate.} A GitHub Action runs a budget checker and fails on regressions.
\begin{lstlisting}[language=yaml]
name: Resource Budget Check
on: [push, pull_request]
jobs:
  budget-check:
    runs-on: ubuntu-latest
    steps:
    - uses: actions/checkout@v2
    - name: Setup Lean
      uses: leanprover/lean-action@v1
    - name: Check Resource Budgets
      run: lake exe check-budgets
\end{lstlisting}

\section{Applications and Value}
We outline domains where unified \emph{correctness $\wedge$ budget} certificates are decisive:
\begin{itemize}[leftmargin=2em]
  \item \textbf{Safety-critical software:} autopilot/medical controllers with deadlines.
  \item \textbf{Blockchain/distributed:} gas-bounded contracts; bounded-communication consensus.
  \item \textbf{Edge and embedded ML:} inference with latency/memory proofs.
  \item \textbf{Education:} teaching algorithmic complexity by types with explicit budgets.
  \item \textbf{Finance/Scientific:} microsecond trading; cluster-allocated simulations.
\end{itemize}

\section{Evaluation Methodology}
We propose unit tests on synthetic examples, case studies with mathlib wrappers, CI stress across merges, and cross-hardware calibration.
\paragraph{Pilot results (illustrative).}
\begin{center}
\begin{tabular}{lccc}
\hline
Algorithm & Static Bound & Measured Max & Ratio \\ \hline
Merge Sort ($n{=}1000$) & 13,500 & 11,234 & 1.20\times \\ 
Binary Search ($n{=}1024$) & 15 & 14 & 1.07\times \\ \hline
\end{tabular}
\end{center}
\paragraph{Limitations.} No automatic bound inference; compositional rules may over-approximate constants; no probabilistic/average-case bounds; HITs and full univalence budgeting are future work.

\section{Roadmap: What Lies Ahead}
\subsection{Near-term (0--3 months)}
\begin{enumerate}[leftmargin=2em]
  \item Prove cost-soundness for STLC in Lean and extract a machine-checkable artifact.
  \item Implement $\modal{R}$ as a Lean-level interface with monotonicity lemmas; add bound-composition tactics.
  \item Publish an open repo with CI and mathlib wrappers.
  \item Add a running example (binary search) with executable budget checks.
\end{enumerate}
\subsection{Mid-term (3--9 months)}
\begin{enumerate}[leftmargin=2em]
  \item Extend to dependent types; thread bounds through $\Pi,\Sigma,\Id$; prototype resource-indexed universes.
  \item Pilot ``feasible univalence'' in a truncated/groupoidal presheaf model and gate usage behind explicit budgets.
  \item Integrate optional plugins for Coq/Isabelle and compare ergonomics.
\end{enumerate}
\subsection{Longer-term}
\begin{enumerate}[leftmargin=2em]
  \item Certified compilation preserving bounds end-to-end.
  \item Advanced cost models: caches, parallelism, I/O, energy; probabilistic bounds; coinduction with rates.
  \item Standardization and outreach: tutorials, benchmarks, adoption guides for industry.
\end{enumerate}

\section{Related Work}
RB-HoTT intersects quantitative/linear types, coeffects, graded/modal type theory, step-indexed models, cost/resource analysis, separation logic, HoTT/cubical, and proof-carrying code/verified compilation.
\paragraph{Comparison with QTT.} QTT tracks usage multiplicities; RB-HoTT targets execution bounds and budgets.
\paragraph{Comparison with Liquid types.} Refinement type systems can encode bounds but lack a compositional feasibility modality and HoTT-oriented semantics within a prover.
\paragraph{Comparison with RAML.} RAML infers polynomial bounds automatically; RB-HoTT focuses on proof-carrying certification inside a proof assistant.

\section{Conclusion}
RB-HoTT reframes foundations around \emph{proof-carrying resources}: artifacts that are both correct and guaranteed to fit within explicit budgets. By embedding budgets into the judgmental fabric, validating them in presheaf semantics (with $\Box$ as a graded comonad), and connecting them to everyday proof engineering, we make boundedness a routine, checkable component of formal development.

\end{document}
